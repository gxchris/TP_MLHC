\documentclass[twoside,11pt]{article}

% Any additional packages needed should be included after jmlr2e.
% Note that jmlr2e.sty includes epsfig, amssymb, natbib and graphicx,
% and defines many common macros, such as 'proof' and 'example'.
%
% It also sets the bibliographystyle to plainnat; for more information on
% natbib citation styles, see the natbib documentation, a copy of which
% is archived at http://www.jmlr.org/format/natbib.pdf

\usepackage{jmlr2e}
%\usepackage{parskip}

% Definitions of handy macros can go here
\newcommand{\dataset}{{\cal D}}
\newcommand{\fracpartial}[2]{\frac{\partial #1}{\partial  #2}}
% Heading arguments are {volume}{year}{pages}{submitted}{published}{author-full-names}

% Short headings should be running head and authors last names
\ShortHeadings{95-845: MLHC Article}{Lastname and Lastname}
\firstpageno{1}

\begin{document}

\title{Heinz 95-845: Article Title on \\Machine Learning and Health Care}

\author{\name Firstname Lastname \email ANDREWID/email@address.edu \\
       \addr Heinz College\\
       Carnegie Mellon University\\
       Pittsburgh, PA, United States
       \AND
       \name Firstname Lastname \email ANDREWID/email@address.edu \\
       \addr Department of Health Research\\
       Another University \\
       City, State, Country} 

\maketitle

\begin{abstract}
  The abstract is the summary of the article. Your potential readers will glance at the abstract to decide
  if the article is worth reading. Make it good--this is your most-read text!  
\end{abstract}

\section{Project overview -- for reference -- remove for submission}

The project will be conducted in March and April, with the \textbf{code and paper due on April 26th}. The secondary review will be due on May 3rd.

\subsection{Grading}
The project is worth 60 points total: proposal 10 points, code 10 points, paper 30 points, and secondary 10 points.

\textbf{Proposal -} Already complete, see proposal TeX.

\textbf{Code -} Your code should be commented and organized. You should be able to hand your code to your secondary and they should be able to run it, and the code should be clear enough to modify it. That may mean you need to provide basic details about how to run your code, \emph{e.g.} in a README file. Your code should be submitted in the form of a git repository (private or public is your choice) and should not contain any private data. 

\textbf{Paper -} Your paper will be assessed for clarity, completeness, and conciseness of the Sections in this template. Your analysis will be assessed for largely for appropriate study design, implementation, and reporting of results. Other Sections in the template will also be graded, e.g. is the task appropriately motivated, are appropriate related works identified, do the conclusions faithfully reflect the results of the analysis, etc. \textbf{The main text of the paper should be 6 to 8 pages, single spaced, not counting References or Appendices; any additional material should go into an Appendix.}

\emph{Study design -} Likely you will not be able to control the actual study design, because the project will be based on a study already conducted or data set already collected. Nonetheless, documentation and review of the design steps is important to assess the study for correctness, impact, etc. Your particular analysis also has elements of a study design and those should be clearly described.

\emph{Implementation -} The steps of the implementation should be documented. Here are some questions you may want to consider. What was the form of the raw data? What data cleaning did you conduct? How did you conduct it? Was there missing data and if so how did you approach those features? Did you conduct cross-validation? Which models did you select for the study (e.g. random forests)? How did you select the parameters of your model? Did you hold out a final test set? What was your a priori hypothesis? How did you plan to test it? What are your outcomes of interest? Why are those medically appropriate outcomes to evaluate? This list is incomplete, but remember to include details pertinent to the scientific question and your approach to answering the question. Those details should be for sufficient for your secondary to match your code to your write-up. If there are many such details, consider moving them to an Appendix.

\emph{Results -} Your report of results should include baseline characteristics, primary and secondary endpoints and visuals to illustrate the outcomes fo your studies. See Section \ref{results} for details.

Please use this template for your write-up; you will find helpful information about the content of each of those sections. Feel free to change/re-order the sections below as you see fit. Please turn in your TeX file and your PDF.

\textbf{Review -} In the last week of class, you (as the secondary) will work with the primary to understand their project, code, and findings. Then you will write up a summary of the primary work and provide constructive feedback. This feedback will be reviewed by myself and shared with the primary. Constructive feedback highlights the strengths and weaknesses of the work as it stands. For weaknesses, your job is to characterize the limitation or concern and provide ways to remedy or mitigate it. It will be assessed in terms of presentation, substantiveness, and on your ability to express scientific assessment. For individual projects, this will be a self-assessment exercise and worth 5 points, with the main paper being 35 points. Please turn in your write-up as a TeX file. I will provide more details as we approach the week of review.


\section{Introduction}
In your own words, tell your audicence about the problem. For example:

``Recent advances in \{subtype of machine learning\} \citep{cite1} have resulted in substantial progress in \{health care domain\}. However, \{describe limitation\} \dots.
In particular, \citet{cite2} describes a novel technique with the potential to improve care.
In this work, we \dots''

Keep in mind who your intended audiences are, e.g., computer scientists, clinicians, statisticians.
In the later sections, you might use some terminology that some reader may not be familiar with.
That's okay. But the introduction should be approachable from a wider audience.

Be sure to motivate and describe this paper's contribution to the literature.
In order to do this, you must summarize existing approaches and/or fields (almost certainly multiple ones),
and describe in what ways your approach should be an improvement. 

Let me re-emphasize: the introduction discusses how your work builds upon and relates to the literature. While you had to be brief in your abstract, the introduction is the place to describe why this paper contributes substantively to the field. Convince me.

At the end of the Introduction, provide the layout of your paper, e.g., ``In Section 2, we provide background on \{subject\}, \dots'',  et cetera.

\section{Background} \label{background}
Background is a presentation of the underlying concepts necessary to understand the details of your approach. Often times there are subsections for key math concepts. For application papers this section often provides the health context that your research addresses. Figures are helpful.

\section{Your Model Name Here} \label{model}
This section describes your model and references the notation you introduced in the Background Section. \textbf{Figures are definitely helpful here}, so that someone who is in your area can visualize how your approach is novel, and someone who is not in your area can visualize what you are doing.

If you introduce new mathematical or statistical methods, use the terminology you defined in Section \ref{background} and define your model. Give the technical details and remember: do be precise and do be concise.

If you are combining existing methods, then you don't need to provide a ton of detail: feel free to just cite other packages and papers and tell us how you put them together.  

If you developed new code that does not (should not) contain sensitive or private information, include a reference, e.g.:

`` Code is available at \url{http://my.github.page.com} ''

\section{Experimental Setup} \label{experiment}
\emph{Note: if the paper is more about the application than the method, this Section may be entitled Methods and appear before Section \ref{model}}

By reading the Experimental Setup, your reader should have the information necessary to replicate the study.

Describe the cohort/data. Provide information about the population, the inclusion and exclusion criteria, what data were extracted, how features were processed, etc. In fact, you may want the following headings. \textbf{A flow chart can be very helpful} to illustrate the experimental setup, study design, inclusion/exclusion process, etc.

For more clinical application papers, each of the sections above might be several paragraphs or pages because we really want to understand the setting.

\subsection{Cohort Selection} 
Describe how the samples you used were selected to form your cohort and also to provide cohort descriptive statistics. In methodologic papers, the ``Table 1'' describing the population by covariate summary statistics goes here. In application papers, ``Table 1'' leads the Results Section. Relevant information about the study design, such how cases and controls were identified, goes here. See Section \ref{results} for an example of how to build a table in LaTeX.

\subsection{Data Extraction} 
Describe the pipeline from raw data to processed data. Figures can be helpful. What assumptions did you make? How did you deal with missing data? Do not place interpretations here except possibly for short justification phrases. Longer discussions about the assumption you made go in the Discussion Section.

\subsection{Feature Choices} 
What features were used? What conversions were necessary? What assumptions (e.g. i.i.d.) are made? with how you might have converted the raw data into features that were used in your algorithm. 

\subsection{Comparison Methods}
To evaluate your model, often times you will compare against existing models.
If so, include them here with a brief description, citation, and any tweaks you made for your experiment.

\subsection{Evaluation Criteria}
Evaluation methods belong here as well.
Perhaps you used accuracy and the AUROC--explain why these are most useful measures of the outcome.

\section{Results} \label{results}

Present the results here.
Do not describe how the results were obtained.
Those descriptions belong in Section \ref{experiment}.

Typically there are multiple parts and subparts of your study.
Use subsections to report the results.

\subsection{Results on Application A} 

Give us some numbers about how well your method works, especially in comparison to some baselines.
You should provide a summary of the results in the text, as well as in tables (such as table~\ref{tab:example}) and figures (such as figure~\ref{fig:example}).  

You may use subfigures/wrapfigures (LaTeX packages) so that figures don't have to span the whole page or multiple figures are side by side.

\begin{table}[htbp]
  \centering 
  \begin{tabular}{lclc} 
    Method & Outcome (\%) \\ 
    \hline \\[-11pt]
    Us & 20.1 \\ 
    Baseline & 18.2 \\ \hline 
  \end{tabular}
  \label{tab:example} 
    \caption{Outcome by method used. These are our results.} 
\end{table}

\begin{figure}[htbp]
  \centering 
  \includegraphics[width=1.5in]{smile.jpeg} 
  \caption{Example smile graphic.}
  \label{fig:example} 
\end{figure} 

\subsection{Results on Application B} 

Did more than one experiment type?

\section{Discussion and Related Work} 

This is where you characterize the outcomes of your method and draw conclusions from you experiment.
The discussion will build upon the Introduction and the Results sections to synthesize where your contribution brings the field. Discuss any implications of your work. 
Discuss limitations of your work.
Are there situations where you should and should not use your method.
What implications are there on policy making, clinical decision making, or future research activities?
Remember to contextualize your work with respect to related work and provide references.

\section{Conclusion} 
Summarize your work one more time, this time assuming the reader has read your paper.
Build suspense for what your next extension to this method would be.

% ACKNOWLEDGEMENTS ONLY GO IN THE CAMERA-READY, NOT THE SUBMISSION
% \acks{Many thanks to all collaborators and funders!}

\bibliography{sample}

\appendix
\section*{Appendix A.}
Some more details about those methods, so we can actually reproduce them.

\end{document}
